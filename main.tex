%-----------------------------------------------------------------
% Documento escrito por Fábio Dias (IST) e Beatriz Vieira (ISEG)
%-----------------------------------------------------------------
\documentclass[12pt]{article}

%-------------------------
% PREÂMBULO
%-------------------------
%-----------------------------------------------------------------
% Documento escrito por Fábio Dias (IST) e Beatriz Vieira (ISEG)
%-----------------------------------------------------------------
\setmainfont{Times New Roman}
%Isto é para definir o tipo de letra
\setmathtt{Arial.ttf}
%Isto é para definir o tipo de letra em modo matemático
\sisetup{group-minimum-digits = 100}
\sisetup{output-decimal-marker = {,}}
%Isto é para o comando \SI{}{}
\setmarginsrb{2 cm}{1 cm}{2 cm}{2 cm}{0.5 cm}{0.5 cm}{1.5 cm}{1 cm}
%Isto é para definir os tamanhos das margens
\onehalfspacing
%Distancias entre rodapé e cabeçalho

\fancypagestyle{date}{\renewcommand{\headrulewidth}{0pt}\fancyhf{}\fancyfoot[C]{\today}}
%Isto é para pôr a data automaticamente na capa

\fancypagestyle{paging}{\renewcommand{\headrulewidth}{0pt}\fancyhf{}\fancyfoot[C]{\footnotesize Página \thepage}}
%Isto é para definir o estilo da paginação do documento e a sua localização na página

\addto\captionsportuguese{\renewcommand{\contentsname}{\sffamily{Índice}}}
%Isto é para mudar o título do índice
\addto\captionsportuguese{\renewcommand{\refname}{}}
%Isto é para alterar o título "Referências" que aparece originalmente associada a \begin{thebibliography}, em branco omite o título

\renewcommand\thesection{\Roman{section}}
\renewcommand\thesubsection{\thesection.\Roman{subsection}}
\renewcommand\thesubsubsection{\thesection.\thesubsection.\Roman{subsubsection}}
%Definição da numeração das sections e subsections, ver mais aqui https://tex.stackexchange.com/questions/3177/how-to-change-the-numbering-of-part-chapter-section-to-alphabetical-r

\titlespacing\section{0pt}{1cm}{0.423cm}
\titleformat{\section}{\normalfont\bfseries\fontsize{14}{6}\sffamily}{\thesection. }{0pt}{}
%Isto é para os títulos das secções do trabalho
\titlespacing\subsection{0pt}{1cm}{0.423cm}
\titleformat{\subsection}{\normalfont\bfseries\fontsize{14}{6}\sffamily}{\thesubsection. }{0pt}{}
%Isto é para os títulos das subsecções do trabalho
\titlespacing\subsubsection{0pt}{1cm}{0.423cm}
\titleformat{\subsubsection}{\normalfont\bfseries\fontsize{14}{6}\sffamily}{\thesubsubsection. }{0pt}{}
%Isto é para os títulos das subsubsecções do trabalho

\captionsetup{labelsep=endash}
%Isto define que as \caption fiquem do tipo "Fig. X - <caption>"
\addto\captionsportuguese{\renewcommand{\figurename}{Fig.}}
%Estipula a legenda das figuras como sendo "Fig. X"
\addto\captionsportuguese{\renewcommand{\tablename}{Tab.}}
%Estipula a legenda das tabelas como sendo "Tab. X"

\DeclareUrlCommand{\bulurl}{\hspace{2mm}\footnotesize\def\UrlFont{\color{blue}\ulined}}\useunder{\uline}{\ulined}{}
%Criação de um comando para meter o link com hyperligação e a azul

%-------------------------
% DOCUMENTO
%-------------------------
\begin{document}

%-------------------------
% CAPA
%-------------------------
\include{titlepage}

%\includepdf[pages=-, scale=2,width=\textwidth]{exemplo.pdf}
%Como incluir um pdf

%-------------------------
\pagestyle{paging}
%Isto é para que as páginas passem a ter paginação

%-------------------------
% INDÍCE
%-------------------------
\tableofcontents
%Isto é efetivamente o que cria o índice
\pagebreak

%-------------------------
% CORPO DO TRABALHO
%-------------------------
\section*{Introdução}
%Como fazer sections que não são numeradas nem contabilizadas no índice

\section{Desenvolvimento}
%%-----------------------------------------------------------------
% Documento escrito por Fábio Dias (IST) e Beatriz Vieira (ISEG)
%-----------------------------------------------------------------

%Copiar para colocar uma equação
%-----------------------------------
\vspace*{-0.3cm}
\begin{equation}
    \text{ENOB} = \frac{\text{SINAD} - 1,76}{6,02} \approx 10,046\:\text{bits}
    \label{eq:exemplo}
\end{equation}
%-----------------------------------


%Copiar para fazer uma tabela
%-----------------------------------
\begin{table}[H]
    \centering
    \caption{Título.}
    \label{tab:exemplo}
    \begin{tabular}{|c|c|c|c|c|}
    \hline
    \cellcolor[HTML]{EFEFEF} Frequência (kHz)   &  \cellcolor[HTML]{EFEFEF} Valor eficaz (V) & \cellcolor[HTML]{EFEFEF} Valor médio (mV) &\cellcolor[HTML]{EFEFEF} $\Delta t$ (ms) & \cellcolor[HTML]{EFEFEF} $\Delta f$ (Hz) \\ \hline
     1,70     & 3,99 & -14,3 & 0,147 & 1,7    \\ \hline
    \end{tabular}
\end{table}
%-----------------------------------


%Copiar para colocar uma imagem
%-----------------------------------
\begin{figure}[H]
    \centering
    \includegraphics[scale = 0.6]{imagens/ist.png} 
    \caption{Título.}
    \label{fig:exemplo.1}
\end{figure}
\vspace*{-0.2cm}
%-----------------------------------


%Copiar para colocar 2 imagens
%-----------------------------------
\begin{figure}[H]
    \centering
    \subfloat[][Título a.]{\label{fig:exemplo.2.a}\includegraphics[width=.4\textwidth]{imagens/ist.png}}\qquad
    \subfloat[][Título b.]{\label{fig:exemplo.2.b}\includegraphics[width=.4\textwidth]{imagens/ist.png}}
    \caption{Título.}
    \label{fig:exemplo.2}
\end{figure}
\vspace*{-0.2cm}
%-----------------------------------


%Copiar para colocar 4 imagens    
%-----------------------------------
\begin{figure}[H]
    \centering
    \subfloat[Título a.]{\label{fig:exemplo.3.a}\includegraphics[width=.4\textwidth]{imagens/ist.png}}\qquad
    \subfloat[Título b.]{\label{fig:exemplo.3.b}\includegraphics[width=.4\textwidth]{imagens/ist.png}}\\
    \subfloat[Título c.]{\label{fig:exemplo.3.c}\includegraphics[width=.4\textwidth]{imagens/ist.png}}\qquad
    \subfloat[Título d.]{\label{fig:exemplo.3.d}\includegraphics[width=.4\textwidth]{imagens/ist.png}}
    \caption{Título.}
    \label{fig:exemplo.3}
\end{figure}
\vspace*{-0.2cm}
%-----------------------------------


%Copiar para colocar uma imagem que fique com texto à volta
%-----------------------------------
\begin{wrapfigure}{R}{0.35\textwidth}
    \centering
    \includegraphics[width=0.3\textwidth]{imagens/ist.png}
    \caption{Título.}
    \label{fig:exemplo.4}
\end{wrapfigure}
%-----------------------------------


%Copiar para colocar Código MatLab
%-----------------------------------
\begin{lstlisting}[frame=none,language=Matlab,label={code:exemplo},firstnumber=1,backgroundcolor=\color{gray!15}, basicstyle=\tiny\ttfamily]
(...)
    for k = 1:length(X(1,:)-1)
        X(:,k+1) = (X(:,k*h)+ h*(Matriz*([X(1,k);X(2,k)])));
        X(1,k+1) = X(1,k+1) + d(k);
    end
(...)
\end{lstlisting}
%-----------------------------------

\subsection{Subsecção}

\subsubsection{Subsubsecção}

\section{Conclusão}

%Truque para ter bibliografia no indice
\section{Bibliografia e Webgrafia}
\vspace*{-0.7cm}
\begin{thebibliography}{10}
%Isto começa a bibliografia e o "10" deve ser alterado pelo numero de referências usadas

%Bibitem é para criar um tópico novo nas referências
\bibitem{1}
Apresentações da Unidade Curricular Instrumentação e Medidas, Pedro M. Ramos, 2021 IST

\bibitem{2}
MATLAB, MathWork\bulurl{https://bit.ly/3a3fAbP}

\end{thebibliography}

\section{Anexos}

\end{document}
