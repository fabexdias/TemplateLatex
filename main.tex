%----------------------------------
% Documento escrito por Fábio Dias
%----------------------------------
\documentclass[12pt]{article}
%-------------------------
% PACKAGES
%-------------------------
\usepackage[table,xcdraw]{xcolor}
\usepackage[portuguese]{babel}
\usepackage[normalem]{ulem}
\usepackage{anyfontsize}
\usepackage{mathtools}
\usepackage{setspace}
\usepackage{hyperref}
\usepackage{ragged2e}
\usepackage{graphicx}
\usepackage{titlesec}
\usepackage{fancyhdr}
\usepackage{listings}
\usepackage{fontspec}
\usepackage{caption}
\usepackage{vmargin}
\usepackage{tocloft}
\usepackage{wrapfig}
\usepackage{amsmath}
\usepackage{amssymb}
\usepackage{siunitx}
\usepackage{parskip}
\usepackage{lipsum}
\usepackage{subfig}
\usepackage{float}
\usepackage{xfrac}
\usepackage{tikz}
\usepackage{url}

%-------------------------
% PREÂMBULO
%-------------------------
%-----------------------------------------------------------------
% Documento escrito por Fábio Dias (IST) e Beatriz Vieira (ISEG)
%-----------------------------------------------------------------
\setmainfont{Times New Roman}
%Isto é para definir o tipo de letra
\setmathtt{Arial.ttf}
%Isto é para definir o tipo de letra em modo matemático
\sisetup{group-minimum-digits = 100}
\sisetup{output-decimal-marker = {,}}
%Isto é para o comando \SI{}{}
\setmarginsrb{2 cm}{1 cm}{2 cm}{2 cm}{0.5 cm}{0.5 cm}{1.5 cm}{1 cm}
%Isto é para definir os tamanhos das margens
\onehalfspacing
%Distancias entre rodapé e cabeçalho

\fancypagestyle{date}{\renewcommand{\headrulewidth}{0pt}\fancyhf{}\fancyfoot[C]{\today}}
%Isto é para pôr a data automaticamente na capa

\fancypagestyle{paging}{\renewcommand{\headrulewidth}{0pt}\fancyhf{}\fancyfoot[C]{\footnotesize Página \thepage}}
%Isto é para definir o estilo da paginação do documento e a sua localização na página

\addto\captionsportuguese{\renewcommand{\contentsname}{\sffamily{Índice}}}
%Isto é para mudar o título do índice
\addto\captionsportuguese{\renewcommand{\refname}{}}
%Isto é para alterar o título "Referências" que aparece originalmente associada a \begin{thebibliography}, em branco omite o título

\renewcommand\thesection{\Roman{section}}
\renewcommand\thesubsection{\thesection.\Roman{subsection}}
\renewcommand\thesubsubsection{\thesection.\thesubsection.\Roman{subsubsection}}
%Definição da numeração das sections e subsections, ver mais aqui https://tex.stackexchange.com/questions/3177/how-to-change-the-numbering-of-part-chapter-section-to-alphabetical-r

\titlespacing\section{0pt}{1cm}{0.423cm}
\titleformat{\section}{\normalfont\bfseries\fontsize{14}{6}\sffamily}{\thesection. }{0pt}{}
%Isto é para os títulos das secções do trabalho
\titlespacing\subsection{0pt}{1cm}{0.423cm}
\titleformat{\subsection}{\normalfont\bfseries\fontsize{14}{6}\sffamily}{\thesubsection. }{0pt}{}
%Isto é para os títulos das subsecções do trabalho
\titlespacing\subsubsection{0pt}{1cm}{0.423cm}
\titleformat{\subsubsection}{\normalfont\bfseries\fontsize{14}{6}\sffamily}{\thesubsubsection. }{0pt}{}
%Isto é para os títulos das subsubsecções do trabalho

\captionsetup{labelsep=endash}
%Isto define que as \caption fiquem do tipo "Fig. X - <caption>"
\addto\captionsportuguese{\renewcommand{\figurename}{Fig.}}
%Estipula a legenda das figuras como sendo "Fig. X"
\addto\captionsportuguese{\renewcommand{\tablename}{Tab.}}
%Estipula a legenda das tabelas como sendo "Tab. X"

\DeclareUrlCommand{\bulurl}{\hspace{2mm}\footnotesize\def\UrlFont{\color{blue}\ulined}}\useunder{\uline}{\ulined}{}
%Criação de um comando para meter o link com hyperligação e a azul

%-------------------------
% DOCUMENTO
%-------------------------
\begin{document}

%-------------------------
% CAPA
%-------------------------
\begin{titlepage}
    \setmainfont{Arial}
    \thispagestyle{date}
    \flushleft
    \includegraphics[scale = 0.3]{imagens/ist.png}\\[1.5 cm]
    \centering
    \textsc{\fontsize{18}{22}\textbf{INSTITUTO SUPERIOR TÉCNICO}}\\[2.6cm]
    \textsc{\fontsize{22}{27}\textbf{U.C.}}\\[1.5 cm]
    \textsc{\fontsize{23}{27}\textbf{TÍTULO}}\\[3.2 cm]
    \textsc{\fontsize{28}{34}\textbf{RELATÓRIO}}\\[5 cm]
    \begin{minipage}{0.5\textwidth}
        \quad
    \end{minipage}
    \setmainfont[Ligatures=TeX]{Arial.ttf}
    \begin{minipage}{0.49\textwidth}
        96XXX -- Aluno\\
        96XXX -- Aluno\\[1cm]
        Grupo XX\\
        Turno de x-feira XXhXX\\
        Professor
    \end{minipage}
\end{titlepage}
%%-----------------------------------------
% Documento adaptado por Fábio Dias (IST)
%-----------------------------------------
\begin{titlepage}
\thispagestyle{date}
\newcommand{\HRule}{\rule{\linewidth}{0.5mm}} % Defines a new command for the horizontal lines, change thickness here

\center % Center everything on the page
\vspace*{1cm} 
%----------------------------------------------------------------------------------------
%	HEADING SECTIONS
%----------------------------------------------------------------------------------------

\textsc{\LARGE Instituto Superior Técnico}\\[1.5cm] % Name of your university/college
\textsc{\Large  TLMXXy}\\[0.5cm] % Major heading such as course name

%----------------------------------------------------------------------------------------
%	TITLE SECTION
%----------------------------------------------------------------------------------------

\HRule \\[0.4cm]
{ \LARGE \bfseries Título}\\[0.4cm] % Subarea
{ \large \bfseries MotoStudent 20XX/XX}\\[0.4cm] % Part
\HRule \\[1.5cm]
 
%----------------------------------------------------------------------------------------
%	AUTHOR SECTION
%----------------------------------------------------------------------------------------

\begin{minipage}{0.4\textwidth}
\begin{flushleft} \large
\emph{Escrito por:}
\par
\textsc{Fábio Dias}\\
\textsc{}


\end{flushleft}
\end{minipage}
\begin{minipage}{0.4\textwidth}
\begin{flushright} \large
\end{flushright}
\end{minipage}\\[3cm]

\begin{center} \large
 
\end{center}
\vspace*{0.35cm}
%----------------------------------------------------------------------------------------
%	LOGO SECTION
%----------------------------------------------------------------------------------------


    \includegraphics[scale=0.09]{imagens/tlmoto.png}\\[1cm]
   

%----------------------------------------------------------------------------------------
\pagebreak
\end{titlepage}

%-------------------------
\pagestyle{paging}
%Isto é para as páginas passem a ter paginação

%-------------------------
% INDÍCE
%-------------------------
\renewcommand{\cftsecleader}{\cftdotfill{\cftdotsep}}
\renewcommand{\cftsecpagefont}{\normalfont}
\renewcommand{\cftsecfont}{\normalfont}
%Isto põe os pontinhos no índice e tira os negritos
\tableofcontents 
%Isto é efetivamente o que queria o índice
\pagebreak

%-------------------------
% CORPO DO TRABALHO
%-------------------------
\section{Introdução}

\section{Desenvolvimento}

\subsection{Subsecção}

\section{Conclusão}

%Truque para ter bibliografia no indice
\section{Bibliografia e Webgrafia}
\vspace*{-0.6cm}
\begin{thebibliography}{10}
%Isto começa a bibliografia e o "10" deve ser alterado pelo numero de referências usadas

%Bibitem é para criar um tópico novo nas referências
\bibitem{1}
Apresentações da Unidade Curricular Instrumentação e Medidas, Pedro M. Ramos, 2021 IST

\bibitem{2}
MATLAB, MathWork\bulurl{https://bit.ly/3EE104b}

\end{thebibliography}

\section{Anexos}

\end{document}
