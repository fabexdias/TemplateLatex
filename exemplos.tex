%-----------------------------------------------------------------
% Documento escrito por Fábio Dias (IST) e Beatriz Vieira (ISEG)
%-----------------------------------------------------------------

%Copiar para colocar uma equação
%-----------------------------------
\vspace*{-0.3cm}
\begin{equation}
    \text{ENOB} = \frac{\text{SINAD} - 1,76}{6,02} \approx 10,046\:\text{bits}
    \label{eq:exemplo}
\end{equation}
%-----------------------------------


%Copiar para fazer uma tabela
%-----------------------------------
\begin{table}[H]
    \centering
    \caption{Título.}
    \label{tab:exemplo}
    \begin{tabular}{|c|c|c|c|c|}
    \hline
    \cellcolor[HTML]{EFEFEF} Frequência (kHz)   &  \cellcolor[HTML]{EFEFEF} Valor eficaz (V) & \cellcolor[HTML]{EFEFEF} Valor médio (mV) &\cellcolor[HTML]{EFEFEF} $\Delta t$ (ms) & \cellcolor[HTML]{EFEFEF} $\Delta f$ (Hz) \\ \hline
     1,70     & 3,99 & -14,3 & 0,147 & 1,7    \\ \hline
    \end{tabular}
\end{table}
%-----------------------------------


%Copiar para colocar uma imagem
%-----------------------------------
\begin{figure}[H]
    \centering
    \includegraphics[scale = 0.6]{imagens/ist.png} 
    \caption{Título.}
    \label{fig:exemplo.1}
\end{figure}
\vspace*{-0.2cm}
%-----------------------------------


%Copiar para colocar 2 imagens
%-----------------------------------
\begin{figure}[H]
    \centering
    \subfloat[][Título a.]{\label{fig:exemplo.2.a}\includegraphics[width=.4\textwidth]{imagens/ist.png}}\qquad
    \subfloat[][Título b.]{\label{fig:exemplo.2.b}\includegraphics[width=.4\textwidth]{imagens/ist.png}}
    \caption{Título.}
    \label{fig:exemplo.2}
\end{figure}
\vspace*{-0.2cm}
%-----------------------------------


%Copiar para colocar 4 imagens    
%-----------------------------------
\begin{figure}[H]
    \centering
    \subfloat[Título a.]{\label{fig:exemplo.3.a}\includegraphics[width=.4\textwidth]{imagens/ist.png}}\qquad
    \subfloat[Título b.]{\label{fig:exemplo.3.b}\includegraphics[width=.4\textwidth]{imagens/ist.png}}\\
    \subfloat[Título c.]{\label{fig:exemplo.3.c}\includegraphics[width=.4\textwidth]{imagens/ist.png}}\qquad
    \subfloat[Título d.]{\label{fig:exemplo.3.d}\includegraphics[width=.4\textwidth]{imagens/ist.png}}
    \caption{Título.}
    \label{fig:exemplo.3}
\end{figure}
\vspace*{-0.2cm}
%-----------------------------------


%Copiar para colocar uma imagem que fique com texto à volta
%-----------------------------------
\begin{wrapfigure}{R}{0.35\textwidth}
    \centering
    \includegraphics[width=0.3\textwidth]{imagens/ist.png}
    \caption{Título.}
    \label{fig:exemplo.4}
\end{wrapfigure}
%-----------------------------------


%Copiar para colocar Código MatLab
%-----------------------------------
\begin{lstlisting}[frame=none,language=Matlab,label={code:exemplo},firstnumber=1,backgroundcolor=\color{gray!15}, basicstyle=\tiny\ttfamily]
(...)
    for k = 1:length(X(1,:)-1)
        X(:,k+1) = (X(:,k*h)+ h*(Matriz*([X(1,k);X(2,k)])));
        X(1,k+1) = X(1,k+1) + d(k);
    end
(...)
\end{lstlisting}
%-----------------------------------
