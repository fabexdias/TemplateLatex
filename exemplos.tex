%-----------------------------------------------------------------
% Documento escrito por Fábio Dias (IST) e Beatriz Vieira (ISEG)
%-----------------------------------------------------------------

%Copiar para colocar uma equação
%-----------------------------------
\vspace*{-0.3cm}
\begin{equation}
    \text{ENOB} = \frac{\text{SINAD [dB]} - 1,76}{6,02} \approx 10,046\:\text{bits}
    \label{eq:exemplo.1}
\end{equation}
%-----------------------------------


%Copiar para colocar um sistema de equações
%-----------------------------------
\vspace*{-0.3cm}
\begin{equation}
    \begin{cases}
    V_{CC} = R_{C}\left(I_C + I_B\right) + R_F I_B + V_{BEon}\\
    V_{CC} = V_{CE} + R_{C}\left(I_C + I_B\right)\\
    I_C = \beta I_B\\
    I_E = I_C + I_B = \left(\beta + 1\right)I_B
    \end{cases}
    \label{eq:exemplo.2}
\end{equation}
%-----------------------------------


%Copiar para colocar 2 sistemas de equações side-by-side
%-----------------------------------
\vspace*{-0.3cm}
\centering{\noindent\begin{minipage}{.45\linewidth}
\begin{equation}
    \begin{cases}
    h_{11A} = \frac{V_1}{I_1}_{V_2 = 0} = R_{in}\\
    h_{12A} = \frac{V_1}{V_2}_{I_1 = 0} = 0\\
    h_{21A} = \frac{I_2}{I_1}_{V_2 = 0} = \frac{-AR_{in}}{R_{out}+r_{\pi}}\\
    h_{22A} = \frac{I_2}{V_2}_{I_1 = 0} = \frac{1}{R_{out} + r_{\pi}} + \frac{1}{r_o}
    \end{cases}
    \label{eq:exemplo.3}
\end{equation}
\end{minipage}
\begin{minipage}{.45\linewidth}
\begin{equation}
    \begin{cases}
    h_{11\beta} = \frac{V_1}{I_1}_{V_2 = 0} = R_1 // R_2\\
    h_{12\beta} = \frac{V_1}{V_2}_{I_1 = 0} = \frac{R_2}{R_2 + R_1}\\
    h_{21\beta} = \frac{I_2}{I_1}_{V_2 = 0} = - \frac{R_2}{R_2 + R_1}\\
    h_{22\beta} = \frac{I_2}{V_2}_{I_1 = 0} = \frac{1}{R_2 + R_1}
    \end{cases}
    \label{eq:exemplo.4}
\end{equation}
\end{minipage}}
%-----------------------------------


%Copiar para fazer uma tabela
%-----------------------------------
\begin{table}[H]
    \caption{Título.}
    \label{tab:exemplo}
    \centering
    \begin{tabular}{c|c|c|c|c|c|}
    \cline{2-6}  
    & \cellcolor[HTML]{EFEFEF} $g_m$ [mS] & \cellcolor[HTML]{EFEFEF} $\beta$ & \cellcolor[HTML]{EFEFEF} $r_\pi$ [\si{\kilo\ohm}] & \cellcolor[HTML]{EFEFEF} $r_0$ [\si{\kilo\ohm}] & \cellcolor[HTML]{EFEFEF} $V_T$ [\si{\milli\volt}] \\ \cline{1-6} 
    \multicolumn{1}{|l|}{\cellcolor[HTML]{EFEFEF} BC547B} & 38,218 & 330 & 8,635 & 33,33 & \multirow{2}{*}{25} \\ \cline{1-5}
    \multicolumn{1}{|l|}{\cellcolor[HTML]{EFEFEF} BC547C} & 40,987 & 580 & 14,151 & 16,67 & \\ \hline
    \end{tabular}
\end{table}
%-----------------------------------


%Copiar para colocar uma imagem
%-----------------------------------
\begin{figure}[H]
    \centering
    \includegraphics[scale = 0.6]{imagens/ist.png} 
    \caption{Título.}
    \label{fig:exemplo.1}
\end{figure}
\vspace*{-0.2cm}
%-----------------------------------


%Copiar para colocar 2 imagens
%-----------------------------------
\begin{figure}[H]
    \centering
    \subfloat[][Título a.]{\label{fig:exemplo.2.a}\includegraphics[width=.4\textwidth]{imagens/ist.png}}\qquad
    \subfloat[][Título b.]{\label{fig:exemplo.2.b}\includegraphics[width=.4\textwidth]{imagens/ist.png}}
    \caption{Título.}
    \label{fig:exemplo.2}
\end{figure}
\vspace*{-0.2cm}
%-----------------------------------


%Copiar para colocar 4 imagens    
%-----------------------------------
\begin{figure}[H]
    \centering
    \subfloat[Título a.]{\label{fig:exemplo.3.a}\includegraphics[width=.4\textwidth]{imagens/ist.png}}\qquad
    \subfloat[Título b.]{\label{fig:exemplo.3.b}\includegraphics[width=.4\textwidth]{imagens/ist.png}}\\
    \subfloat[Título c.]{\label{fig:exemplo.3.c}\includegraphics[width=.4\textwidth]{imagens/ist.png}}\qquad
    \subfloat[Título d.]{\label{fig:exemplo.3.d}\includegraphics[width=.4\textwidth]{imagens/ist.png}}
    \caption{Título.}
    \label{fig:exemplo.3}
\end{figure}
\vspace*{-0.2cm}
%-----------------------------------


%Copiar para colocar uma imagem que fique com texto à volta
%-----------------------------------
\begin{wrapfigure}{R}{0.35\textwidth}
    \centering
    \includegraphics[width=0.3\textwidth]{imagens/ist.png}
    \caption{Título.}
    \label{fig:exemplo.4}
\end{wrapfigure}
%-----------------------------------


%Copiar para colocar Código MatLab
%-----------------------------------
\begin{lstlisting}[frame=none,language=Matlab,label={code:exemplo},firstnumber=1,backgroundcolor=\color{gray!15}, basicstyle=\tiny\ttfamily]
(...)
    for k = 1:length(X(1,:)-1)
        X(:,k+1) = (X(:,k*h)+ h*(Matriz*([X(1,k);X(2,k)])));
        X(1,k+1) = X(1,k+1) + d(k);
    end
(...)
\end{lstlisting}
%-----------------------------------


%Copiar para circuito em circuitikz
%-----------------------------------
\begin{figure}[H]
    \centering{\resizebox{0.45\textwidth}{!}{
    \begin{circuitikz}[american, thick] 
    \draw
	(0,0) node[npn](npn1){$T_1$}
	(npn1.B) to [R,l_=$R_g$,-*] ++(-2,0) node[left](gnd1){$V_g$}
	(npn1.E) -- (0,-1)
	(npn1.C) to [R,l=$R_C$] (0,3) -- (0,2.9) to[short,-*] (0,3) node[above]{\SI{12}{\volt}}
	(0,-1) node[ground]{}
	;
    \end{circuitikz}}}
    \caption{Título.}
    \label{sch:exemplo.1}
\end{figure}
%-----------------------------------


%Copiar para circuito sibe-by-side em circuitikz
%-----------------------------------
\vspace*{-0.2cm}
\begin{figure}[H]
    \centering
    \subfloat[][Título a.]{\label{sch:exemplo.2.a}\resizebox{0.35\textwidth}{!}{\centering{\begin{circuitikz}[american voltages, thick]
    \draw
	(0,0) node[npn](npn1){$T_1$}
	(npn1.B) to [R,l_=$R_g$,-*] ++(-2,0) node[left](gnd1){$V_g$}
	(npn1.E) -- (0,-1)
	(npn1.C) to [R,l=$R_C$] (0,3) -- (0,2.9) to[short,-*] (0,3) node[above]{\SI{12}{\volt}}
	(0,-1) node[ground]{}
	;
    \end{circuitikz}}}}\qquad
    \subfloat[][Título b.]{\label{sch:exemplo.2.b}\resizebox{0.35\textwidth}{!}{\centering{
    \begin{circuitikz}[american voltages, thick]
    \draw
	(0,0) node[npn](npn1){$T_1$}
	(npn1.B) to [R,l_=$R_g$,-*] ++(-2,0) node[left](gnd1){$V_g$}
	(npn1.E) -- (0,-1)
	(npn1.C) to [R,l=$R_C$] (0,3) -- (0,2.9) to[short,-*] (0,3) node[above]{\SI{12}{\volt}}
	(0,-1) node[ground]{}
	;
    \end{circuitikz}}}}
    \caption{Título.}
    \label{sch:exemplo.2}
\end{figure}
%-----------------------------------